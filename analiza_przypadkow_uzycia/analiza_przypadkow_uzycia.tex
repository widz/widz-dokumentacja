\documentclass[12pt,leqno,twoside]{mwart}
\usepackage[utf8x]{inputenc}
\usepackage{polski}
\usepackage{a4wide}

%---------------------------------------------------------------------------------
%paginy!
\usepackage{fancyhdr}
\pagestyle{fancy}
\fancyhead{}
\renewcommand{\sectionmark}[1]{\markright{\thesection.\ #1. }}
\fancyhead[LE]{WiDz -- Wirtualny Dziennik. Analiza przypadków użycia}
\fancyhead[RO]{\rightmark}
\fancyfoot{} % clear all footer fields
\fancyfoot[LE,RO]{}
\fancyfoot[CE]{\thepage}
\fancyfoot[CO]{\thepage}
\addtolength{\headheight}{1.5pt} % pionowy odstep na kreske
\renewcommand{\headrulewidth}{0.4pt}
\renewcommand{\footrulewidth}{0.0pt}
%--------------------------------------------------------------------------------

\makeatletter
\renewcommand{\@biblabel}[1]{\quad #1.}
\makeatother

\renewcommand{\figurename}{Rys.}
\renewcommand{\labelitemi}{-}

\begin{document}

\begin{titlepage}
\begin{center}
Instytut Informatyki Uniwersytetu Wrocławskiego \\
Studencka Pracownia Inżynierii Oprogramowania, G4 \\
\vspace{4cm}
\Large Michał Kopacz, Mateusz Nahalewicz, Karol Bajko \\
\vspace{0.5cm}
\huge Dokumentacja projektu \mbox{\textbf{WiDz -- Wirtualny Dziennik}} \\ \Large Analiza przypadków użycia\\
\vspace{1cm}
\normalsize Wersja 1.1
\vfill
\normalsize Wrocław 2009
\end{center}
\end{titlepage}

\newpage

\begin{table}
	\centering
	\caption{Historia zmian w~dokumencie}
		\begin{tabular}{|r|c|c|l|l|}
		\hline
		Lp. 	& Data       & Nr wersji 	& Autorzy           		& Zmiana \\ \hline
		1.   	& 2009-12-05 & 1.0       	& Mateusz Nahalewicz & Utworzenie dokumentu \\ \hline
		2.   	& 2009-12-09 & 1.1       	& Michał Kopacz & Korekta \\ \hline
		\end{tabular}
\end{table}

\newpage

\tableofcontents

\newpage


\section{Wstęp}
\noindent Przypadek użycia to opis sekwencji działań mającej określony cel. W trakcie jego realizacji zachodzi interakcja pomiędzy użytkownikiem a aplikacją. Celem specyfikowania przypadków użycia jest ustalenie przebiegu procesów oraz wynikających z nich wymagań w formie zrozumiałej jednocześnie w całości dla użytkownika i analizującego wymagania oraz w dużej części dla wykonawcy produktu. Przypadki użycia opisują system z punktu widzenia użytkownika. Dzięki nim użytkownikowi jest łatwiej sobie wyobrazić jak system będzie funkcjonował. Użytkownik nie czyta skomplikowanego opisu, lecz krótki scenariusz, mówiący na przykład o tym jak sprawdzić plan lekcji konkretnego nauczyciela.\\

\section{Analiza przypadków użycia}
\subsection{Rejestracja użytkownika w programie}
\begin{enumerate}
\item Uruchomienie programu.
\item Wybór opcji rejestracji.
\item Wpisanie unikatowego identyfikatora.
\item Wpisanie hasła.
\item Ponowne wpisanie hasła.
\item Uzupełnienie pozostałych danych osobowych.
\item Aktywacja konta przez administratora po okazaniu dokumentu tożsamości.
\end{enumerate}
\subsection{Uwierzytelnienie użytkownika}
\begin{enumerate}
\item Uruchomienie programu.
\item Wpisanie identyfikatora.
\item Wpisanie hasła.
\item Zatwierdzenie operacji.
\end{enumerate}
\subsection{Wpisanie oceny}
\noindent  \textbf{Założenie:} Oceny może wpisywać tylko nauczyciel.\\
\begin{enumerate}
\item Uwierzytelnienie użytkownika.
\item Wybór opcji wpisywania oceny.
\item WiDz wyświetla listę przedmiotów, z których nauczyciel może wystawiać oceny.
\item Wybór przedmiotu.
\item WiDz wyświetla listę klas, w których nauczyciel może wystawiać oceny.
\item Wybór klasy.
\item Wpisanie ocen wybranym uczniom.
\item Zatwierdzenie operacji.
\end{enumerate}
\subsection{Wpisanie nieobecności}
\noindent  \textbf{Założenie:} Nieobecności może wpisywać tylko nauczyciel.\\
\begin{enumerate}
\item Uwierzytelnienie użytkownika.
\item Wybór opcji wpisywania nieobecności.
\item WiDz wyświetla listę przedmiotów, z których nauczyciel może wpisywać nieobecności.
\item Wybór przedmiotu.
\item WiDz wyświetla listę klas, w których nauczyciel może wpisywać nieobecności.
\item Wybór klasy.
\item Wybór daty.
\item Wpisanie nieobecności wybranym uczniom.
\item Zatwierdzenie operacji.
\end{enumerate}
\subsection{Przegląd statystyk}
\noindent \textbf{Założenie:} Statystyki może przeglądać tylko nauczyciel.\\
\begin{enumerate}
\item Uwierzytelnienie użytkownika.
\item Wybór opcji przeglądania statystyk.
\item Wybór przedmiotu.
\item Wybór klasy.
\item WiDz analizuje osiągnięcia uczniów danej klasy z danego przedmiotu.
\item Prezentacja wyników.
\end{enumerate}
\subsection{Przegląd planu lekcji}
\begin{enumerate}
\item Uwierzytelnienie użytkownika.
\item Wybór opcji przeglądania planu lekcji.
\item Użytkownik decyduje czy chce zobaczyć plan konkretnego nauczyciela czy konkretnej klasy.
\item Wybór nauczyciela lub klasy w zależności od poprzedniej decyzji.
\item Prezentacja planu lekcji.
\end{enumerate}
\subsection{Uregulowanie opłat na rzecz szkoły}
\begin{enumerate}
\item Uwierzytelnienie użytkownika.
\item Wybór opcji uregulowania opłat.
\item Opiekun wybiera składki, które chce uregulować.
\item WiDz łączy się z systemem \textit{PayPal}, który umożliwia dokończenie operacji wpłaty.
\end{enumerate}
\subsection{Sprawdzenie ocen przez ucznia}
\begin{enumerate}
\item Uwierzytelnienie użytkownika.
\item Wybór opcji przeglądania ocen.
\item Wybór przedmiotu.
\item WiDz wyświetla listę wszystkich ocen z danego przedmiotu wraz z ich krótkimi opisami.
\end{enumerate}
\subsection{Sprawdzenie nieobecności przez ucznia}
\begin{enumerate}
\item Uwierzytelnienie użytkownika.
\item Wybór opcji przeglądania listy nieobecności.
\item Wybór przedmiotu.
\item WiDz wyświetla listę nieobecności ucznia na lekcjach z wybranego przedmiotu.
\end{enumerate}
\subsection{Udział ucznia w ankiecie}
\begin{enumerate}
\item Uwierzytelnienie użytkownika.
\item Wybór opcji udziału w ankiecie.
\item WiDz wyświetla formularz z pytaniami.
\item Wypełnienie formularza.
\item Zatwierdzenie odpowiedzi.
\end{enumerate}
\subsection{Nadanie przez ucznia wiadomości w kąciku klasowym}
\begin{enumerate}
\item Uwierzytelnienie użytkownika.
\item Wybór opcji przeglądania kącika klasowego.
\item Wpisanie tekstu wiadomości do kącika klasowego.
\item Wysłanie wiadomości.
\end{enumerate}
\subsection{Odczytanie przez ucznia wiadomości z kącika klasowego}
\begin{enumerate}
\item Uwierzytelnienie użytkownika.
\item Wybór opcji przeglądania kącika klasowego.
\item Wyświetlenie najnowszych wiadomości z kącika klasowego.
\end{enumerate}
\subsection{Sprawdzenie ocen przez opiekuna}
\begin{enumerate}
\item Uwierzytelnienie użytkownika.
\item Wybór opcji przeglądania ocen.
\item WiDz wyświetla listę uczniów, których oceny opiekun może przeglądać.
\item Wybór przedmiotu.
\item WiDz wyświetla listę wszystkich ocen danego ucznia z danego przedmiotu wraz z ich krótkimi opisami.
\end{enumerate}
\end{document}
