\documentclass[12pt,leqno,twoside]{mwart}
\usepackage[polish]{babel}
\usepackage[utf8]{inputenc}
\usepackage[T1]{fontenc}
\usepackage{a4wide}

\widowpenalty=10000
\clubpenalty=10000
\raggedbottom

%---------------------------------------------------------------------------------
%paginy!
\usepackage{fancyhdr}
\pagestyle{fancy}
\fancyhead{}
\renewcommand{\sectionmark}[1]{\markright{\thesection.\ #1. }}
\renewcommand{\sectionmark}[1]{\markright{\thesection.\ #1}}
\fancyhead[LE]{WiDz -- Wirtualny Dziennik. Koncepcja ogólna}
\fancyhead[RO]{\rightmark}
\fancyfoot{} % clear all footer fields
\fancyfoot[LE,RO]{}
\fancyfoot[CE]{\thepage}
\fancyfoot[CO]{\thepage}
\addtolength{\headheight}{1.5pt} % pionowy odstep na kreske
\renewcommand{\headrulewidth}{0.4pt}
\renewcommand{\footrulewidth}{0.0pt}
%--------------------------------------------------------------------------------

\makeatletter
\renewcommand{\@biblabel}[1]{\quad #1.}
\makeatother

\renewcommand{\figurename}{Rys.}
\renewcommand{\labelitemi}{-}

\begin{document}

\begin{titlepage}
\begin{center}
Instytut Informatyki Uniwersytetu Wrocławskiego \\
Studencka Pracownia Inżynierii Oprogramowania, G4 \\
\vspace{4cm}
\Large Michał Kopacz, Mateusz Nahalewicz, Karol Bajko \\
\vspace{0.5cm}
\huge Dokumentacja projektu \mbox{\textbf{WiDz -- Wirtualny Dziennik}} \\ \Large Słownik\\
\vspace{1cm}
\normalsize Wersja 1.0
\vfill
\normalsize Wrocław 2009
\end{center}
\end{titlepage}

\newpage

\begin{table}
	\centering
	\caption{Historia zmian w~dokumencie}
		\begin{tabular}{|r|c|c|l|l|}
		\hline
		Lp. 	& Data       & Nr wersji 	& Autor           		& Zmiana \\ \hline
		1.   	& 2009-11-24 & 1.0       	& Karol Bajko & Utworzenie dokumentu \\ \hline
		\end{tabular}
\end{table}

\newpage

\tableofcontents

\newpage

\section{Słownik}
%\noindent Słownik terminów użytych w~dokumencie znajduje się w~\cite{SLO}.
\noindent Poniżej wyjaśniamy wybrane terminy użyte w dokumentacji projektu WiDz.

\begin{description}
\item[API]-- zbiór procedur umożliwiających komunikację programu z~biblioteką bądź serwisem który go udostępnia i~korzystaniu z~ich usług\\
\item[dziennik elektroniczny]-- program komputerowy służący do rejestracji wszelkich informacji o~uczniu i zajęciach\\
\item[frekwencja]-- \\
\item[gilosz]-- \\
\item[HTTPS]-- \\
\item[login]-- \\
\item[ocena opisowa]-- \\
\item[PayPal]-- oprogramowanie umożliwiające dokonywanie bezpiecznych płatności i przelewów w~Internecie\\
\item[PDF]-- \\
\item[Ruby]-- język obiektowy, opracowano w~nim ramę Ruby on Rails\\
\item[Ruby on Rails]-- rama tworzenia aplikacji internetowych w~języku Ruby\\
\item[SSL]-- \\
\item[środowisko ramowe (rama, ang. \textit{framework})]-- program wspomagający tworzenie oprogramowania, jego rozwój i testowanie\\
\item[TSL]-- \\
\item[wiadomość prywatna]-- \\
\end{description}

\end{document}
