\documentclass[12pt,leqno,twoside]{mwart}
\usepackage[polish]{babel}
\usepackage[utf8]{inputenc}
\usepackage[T1]{fontenc}
\usepackage{a4wide}

\widowpenalty=10000
\clubpenalty=10000
\raggedbottom

%---------------------------------------------------------------------------------
%paginy!
\usepackage{fancyhdr}
\pagestyle{fancy}
\fancyhead{}
\renewcommand{\sectionmark}[1]{\markright{\thesection.\ #1. }}
\renewcommand{\sectionmark}[1]{\markright{\thesection.\ #1}}
\fancyhead[LE]{WiDz -- Wirtualny Dziennik. Koncepcja ogólna}
\fancyfoot{} % clear all footer fields
\fancyfoot[LE,RO]{}
\fancyfoot[CE]{\thepage}
\fancyfoot[CO]{\thepage}
\addtolength{\headheight}{1.5pt} % pionowy odstep na kreske
\renewcommand{\headrulewidth}{0.4pt}
\renewcommand{\footrulewidth}{0.0pt}
%--------------------------------------------------------------------------------

\makeatletter
\renewcommand{\@biblabel}[1]{\quad #1.}
\makeatother

\renewcommand{\figurename}{Rys.}
\renewcommand{\labelitemi}{-}

\begin{document}

\begin{titlepage}
\begin{center}
Instytut Informatyki Uniwersytetu Wrocławskiego \\
Studencka Pracownia Inżynierii Oprogramowania, G4 \\
\vspace{4cm}
\Large Michał Kopacz, Mateusz Nahalewicz, Karol Bajko \\
\vspace{0.5cm}
\huge Dokumentacja projektu \mbox{\textbf{WiDz -- Wirtualny Dziennik}} \\ \Large Słownik\\
\vspace{1cm}
\normalsize Wersja 1.1
\vfill
\normalsize Wrocław 2009
\end{center}
\end{titlepage}

\newpage

\begin{table}
	\centering
	\caption{Historia zmian w~dokumencie}
		\begin{tabular}{|r|c|c|l|l|}
		\hline
		Lp. 	& Data       & Nr wersji 	& Autor           		& Zmiana \\ \hline
		1.   	& 2009-11-24 & 1.0       	& Karol Bajko & Utworzenie dokumentu \\ \hline
		2.   	& 2010-01-20 & 1.1       	& Karol Bajko & Korekta \\ \hline
		\end{tabular}
\end{table}

\section{Słownik}
\noindent Poniżej wyjaśniamy wybrane terminy użyte w dokumentacji programu WiDz.

\begin{description}
\item[API]-- zbiór procedur umożliwiających komunikację programu z~biblioteką lub serwisem który je udostępnia, a następnie korzystaniu z~ich usług\\
\item[BDD (ang. \textit{Behavior-Driven Development})]-- technika tworzenia programu opierająca się na pisaniu testów jednostkowych, opisujących konkretne obiekty oraz testów w postaci scenariuszy opisujących zadania, które powinna wykonać aplikacja przed jej właściwym kodowaniem; następnym krokiem jest napisanie fragmentu programu, aby testy przeszły pomyślnie\\
\item[CSRF (ang. \textit{Cross-Site Request Forgery})]-- sposób ataku na serwis internetowy, który polega na wymuszeniu użytkownika mającego dostęp do zamkniętych części serwisu na wykonanie niebezpiecznego kodu\\
\item[CSS (ang. \textit{Cascading Style Sheets})]-- język służący do opisu wyglądu stron WWW\\
\item[dziennik elektroniczny]-- program komputerowy służący do rejestracji wszelkich informacji o~uczniu i zajęciach\\
\item[gilosz]-- grawerowana ozdoba, trudna do podrobienia, wykorzystywana jako tło w dokumentach, np. papierach wartościowych, świadectwach szkolnych\\
\item[HTML (ang. \textit{Hypertext Markup Language})]-- język wykorzystywany do tworzenia stron WWW\\
\item[HTTP (ang. \textit{Hypertext Transfer Protocol})]-- protokół typu klient/serwer dla sieci WWW, pozwala przeglądarkom na dostęp do serwerów i pobieranie dokumentów napisanych w języku HTML\\
\item[HTTP Apache]-- oprogramowanie serwerowe protokołu HTTP\\
\item[HTTPS]-- szyfrowana wersja protokołu HTTP z wykorzystaniem metody szyfrowania danych SSL\\
\item[JavaScript]-- język programowania służący do budowania interaktywnych z użytkownikiem części serwisu WWW\\
\item[OpenSSL]-- biblioteka kryptograficzna, umożliwia wykorzystanie metody szyfrowania danych SSL do komunikacji pomiędzy serwerem WWW a użytkownikiem\\
\item[PayPal]-- oprogramowanie umożliwiające dokonywanie bezpiecznych płatności i przelewów w~Internecie\\
\item[REST (ang. \textit{Representational State Transfer})]-- styl projektowania serwisu WWW, który bazuje na myśli, że sieć WWW to ogromny zbiór zasobów, a sam serwis WWW to program który obserwuje i modyfikuje stan tych zasobów\\
\item[Ruby]-- język obiektowy, opracowano w~nim ramę Ruby on Rails\\
\item[Ruby on Rails]-- rama tworzenia aplikacji internetowych w~języku Ruby\\
\item[SHA-1]-- funkcja skrótu, generująca unikalny ciąg znaków stałej długości dla dowolnie dużej wiadomości, minimalna zmiana w treści wiadomości powoduje wygenerowanie innego ciągu znaków\\
\item[SSL (ang. \textit{Secure Socket Layer})]-- protokół służący do bezpiecznej transmisji danych\\
\item[SQL Injection]-- luka w zabezpieczeniu serwisu internetowego, które umożliwia osobie atakującej na dostęp do zablokowanych danych w bazie danych\\
\item[SZDB MySQL]-- system zarządzania bazą danych MySQL\\
\item[środowisko ramowe (rama, ang. \textit{framework})]-- oprogramowanie wspomagające tworzenie innych programów, ich rozwój i testowanie\\
\item[TSL (ang. \textit{Transport Layer Security})]-- metoda szyfrowania danych przesyłanych pomiędzy serwerem WWW a użytkownikiem; rozwinięta wersja protokołu SSL\\
\item[wiadomość prywatna]-- wiadomość skierowana do konkretnej osoby, żadna inna osoba nie ma wglądu do treści wiadomości\\
\item[XSS (ang. \textit{Cross-Site Scripting})]-- sposób ataku na serwis internetowy, który umożliwia wprowadzenie niebezpiecznego kodu na atakowaną stronę\\
\end{description}

\end{document}
