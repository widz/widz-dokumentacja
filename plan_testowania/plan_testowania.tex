\documentclass[12pt,leqno,twoside]{mwart}
\usepackage[utf8x]{inputenc}
\usepackage{polski}
\usepackage{a4wide}
\usepackage{url}

\widowpenalty=10000
\clubpenalty=10000
\raggedbottom
\makeatletter
\renewcommand*{\@biblabel}[1]{\hfill#1.}
\makeatother

%---------------------------------------------------------------------------------
%paginy!
\usepackage{fancyhdr}
\pagestyle{fancy}
\fancyhead{}
\renewcommand{\sectionmark}[1]{\markright{\thesection.\ #1. }}
\fancyhead[LE]{WiDz -- Wirtualny Dziennik. Plan testowania}
\fancyhead[RO]{\rightmark}
\fancyfoot{} % clear all footer fields
\fancyfoot[LE,RO]{}
\fancyfoot[CE]{\thepage}
\fancyfoot[CO]{\thepage}
\addtolength{\headheight}{1.5pt} % pionowy odstep na kreske
\renewcommand{\headrulewidth}{0.4pt}
\renewcommand{\footrulewidth}{0.0pt}
%--------------------------------------------------------------------------------

%\makeatletter
%\renewcommand{\@biblabel}[1]{\quad #1.}
%\makeatother

\renewcommand{\figurename}{Rys.}
\renewcommand{\labelitemi}{-}

\begin{document}

\begin{titlepage}
\begin{center}
Instytut Informatyki Uniwersytetu Wrocławskiego \\
Studencka Pracownia Inżynierii Oprogramowania, G4 \\
\vspace{4cm}
\Large Michał Kopacz, Mateusz Nahalewicz, Karol Bajko \\
\vspace{0.5cm}
\huge Dokumentacja projektu \mbox{\textbf{WiDz -- Wirtualny Dziennik}} \\ \Large Plan testowania\\
\vspace{1cm}
\normalsize Wersja 1.2
\vfill
\normalsize Wrocław 2010
\end{center}
\end{titlepage}

\newpage

\begin{table}
	\centering
	\caption{Historia zmian w~dokumencie}
		\begin{tabular}{|r|c|c|l|l|}
		\hline
		Lp. 	& Data       & Nr wersji 	& Autorzy           		& Zmiana \\ \hline
		1.   	& 2010-01-10 & 1.0       	& \mbox{Michał Kopacz} & Utworzenie dokumentu \\ \hline
		2.   	& 2010-01-12 & 1.1       	& \mbox{Mateusz Nahalewicz} & Korekta \\ \hline
		3.   	& 2010-01-26 & 1.2       	& \mbox{Michał Kopacz} & Korekta \\ \hline
		\end{tabular}
\end{table}

\newpage

\tableofcontents

\newpage

\section{Wprowadzenie}
\subsection{Cel dokumentu}
\noindent Celem dokumentu jest określenie zakresu testów sprawdzających zgodność programu WiDz ze specyfikacją wymagań opisanych w dokumencie \cite{WYM}, oraz wydajność i bezpieczeństwo podczas używania programu. Testy weryfikują również funkcje poszczególnych części programu i~spójność między nimi. Testowanie ma za zapewnić najwyższą jakość programu WiDz. \\
\subsection{Etapy testów}
\noindent Testowanie programu WiDz przebiega zgodnie z~założeniami modelu V. Zaletą modelu V jest to, że dwa najważniejsze etapy tj. projektowanie i weryfikacja, nie następują po sobie, lecz są wykonywane równolegle. \\
\indent Testy będą planowane przed wykonaniem każdej iteracji budowy programu. Będą obejmować sprawdzenie każdego modułu z~osobna oraz przetestowanie spójności modułów po zakończeniu iteracji. Gotowy program zostanie poddany całościowym testom systemowym po zakończeniu etapu budowy.

\section{Rodzaje testów}
\subsection{Testy jednostkowe}
\noindent Testy jednostkowe programu WiDz są projektowane za pomocą biblioteki RSpec.\footnote{Biblioteka BDD (ang. BDD --- \textit{Behavior-Driven Development}) dla aplikacji języka Ruby, umożliwia pisanie testów jednostkowych, opisujących konkretne obiekty, oraz testów w~postaci scenariuszy opisujących zadania, które powinna wykonać aplikacja.} \\

\subsubsection{Testy brzegowe}
\noindent Testy brzegowe pozwalają wykryć i przetestować obszary, w których prawdopodobieństwo niepoprawnego działania programu WiDz jest największe. Podczas testowania zostaną sprawdzone warunki brzegowe powstałe na skutek: 
\begin{itemize}
	\item wprowadzenia całkowicie błędnych lub niespójnych danych wejściowych, 
	\item wadliwie sformatowanych danych, np. adres e-mail wpisany podczas rejestracji użytkownika w programie WiDz,
	\item pojawienia się wartości przekraczających oczekiwania, np. ocena 7 wpisana do dziennika,
	\item pojawienia się duplikatów na listach, np. podczas wyświetlania listy uczniów danej klasy,
	\item zakłócenia przewidywanego porządku zdarzeń, np. próby wygenerowania raportu przed rozpoczęciem sesji.
\end{itemize} 
\subsection{Testy scalenia}
\subsubsection{Testy funkcjonalności}
\noindent Testy funkcjonalności i~zgodności ze specyfikacją wymagań programu WiDz, są wykonywane w postaci przypadków testowych, opracowanych na podstawie przypadków użycia, określonych w \cite{PU}. \\
\indent Testy te sprawdzają działanie programu z interfejsem użytkownika. Sprawdzana jest reakcja programu zarówno na poprawne, jak i niepoprawne zachowania użytkownika. \\
\subsubsection{Testy wydajnościowe}
\noindent Podczas testów wydajnościowych programu WiDz zostaną zweryfikowane:
\begin{itemize}
	\item czas odpowiedzi programu przy dużym obciążeniu serwera,
	\item czas odpowiedzi bazy danych przy czytaniu dużej ilości danych,
	\item odporność bazy danych na przeciążenie zbyt dużą ilością danych do zapisu.
\end{itemize}
\indent Testy są wykonywane po każdej fazie iteracji, dzięki czemu uzyskuje się informacje o~tym, które elementy programu należy poprawić, aby program działał szybciej i stabilniej. \\
\subsubsection{Testy spójności komponentów}
\noindent Testowanie spójności komponentów umożliwia sprawdzenie, czy przesyłanie danych między komponentami jest poprawne. Sprawdzana jest również spójność danych w bazie danych. \\
\subsubsection{Testy regresywne}
\noindent Testy regresywne są wykonywane po każdym zakończeniu iteracji. Mają sprawdzić poprawne działanie programu po wprowadzeniu zmian i naprawie błędów.\\
\subsection{Testy systemowe}
\subsubsection{Testy bezpieczeństwa}
\noindent Testy bezpieczeństwa mają sprawdzić odporność programu na próby:
\begin{itemize}
	\item włamania do bazy danych metodą wstrzykiwania,
	\item otwarcie sesji użytkownika bez znajomości hasła,
	\item uzyskania przez użytkownika nieprzysługujących mu uprawnień,
	\item zablokowania dostępu do programu przez przeciążenie programu.
\end{itemize}
\subsubsection{Testy dotyczące interfejsu użytkownika}
\noindent Interfejs użytkownika programu WiDz jest sprawdzany pod względem intuicyjnej nawigacji, szybkości dostępu do poszukiwanych informacji oraz zrozumiałych dla użytkownika komunikatów. \\
\subsubsection{Testy konfiguracyjne}
\noindent Podczas testów konfiguracyjnych jest sprawdzane poprawne działanie programu WiDz w~poszczególnych systemach operacyjnych (Linux, Windows, Mac OS), oraz przeglądarkach internetowych (Safari, Firefox, Internet Explorer, Konqueror, Opera, Google Chrome).\\
\subsection{Testy akceptacyjne}
\noindent Testy akceptacyjne są przeprowadzane przez użytkowników programu WiDz. \\
\section{Zasoby}
\subsection{Sprzęt}
\noindent Program WiDz jest testowany z użyciem dwóch serwerów oraz trzech stanowisk klientów. Pierwszy z serwerów spełnia minimalne wymagania sprzętowe, drugi, wymagania zalecane, określone w \cite{WYM}.
\subsection{Oprogramowanie}
\noindent Na każdym stanowisku klienta jest zainstalowany jeden z następujących systemów operacyjnych: Windows, Linux, Mac OS. W zależności od systemu są zainstalowane następujące przeglądarki internetowe: \\
	\indent Windows: Safari, Firefox, Internet Explorer, Opera, Google Chrome, \\
	\indent Linux: Safari, Firefox, Opera, Google Chrome, Konqueror,\\
	\indent Mac OS: Safari, Firefox, Opera, Google Chrome. \
\section{Raporty}
\noindent Błędy wykryte w~czasie testowania będa rejestrowane za pomocą oprogramowania\linebreak Bugzilla\footnote{Oprogramowanie umożliwiające tworzenie serwisu internetowego do raportowania błędów w dowolnym oprogramowaniu. Patrz \url{http://www.bugzilla.org/}}. \\
\begin{thebibliography}{9}
	\bibitem{WYM} Michał Kopacz, Karol Bajko, Mateusz Nahalewicz: {\it Dokumentacja projektu WiDz -- Wirtualny Dziennik. Specyfikacja wymagań}. Wrocław, SPIO IIUWr 2009.
	\bibitem{PU} Michał Kopacz, Karol Bajko, Mateusz Nahalewicz: {\it Dokumentacja projektu WiDz -- Wirtualny Dziennik. Analiza przypadków użycia}. Wrocław, SPIO IIUWr 2009.
\end{thebibliography}
\end{document}