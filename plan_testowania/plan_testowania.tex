\documentclass[12pt,leqno,twoside]{mwart}
\usepackage[utf8x]{inputenc}
\usepackage{polski}
\usepackage{a4wide}
\usepackage{url}

\widowpenalty=10000
\clubpenalty=10000
\raggedbottom

%---------------------------------------------------------------------------------
%paginy!
\usepackage{fancyhdr}
\pagestyle{fancy}
\fancyhead{}
\renewcommand{\sectionmark}[1]{\markright{\thesection.\ #1. }}
\fancyhead[LE]{WiDz -- Wirtualny Dziennik. Specyfikacja wymagań}
\fancyhead[RO]{\rightmark}
\fancyfoot{} % clear all footer fields
\fancyfoot[LE,RO]{}
\fancyfoot[CE]{\thepage}
\fancyfoot[CO]{\thepage}
\addtolength{\headheight}{1.5pt} % pionowy odstep na kreske
\renewcommand{\headrulewidth}{0.4pt}
\renewcommand{\footrulewidth}{0.0pt}
%--------------------------------------------------------------------------------

%\makeatletter
%\renewcommand{\@biblabel}[1]{\quad #1.}
%\makeatother

\renewcommand{\figurename}{Rys.}
\renewcommand{\labelitemi}{-}

\begin{document}

\begin{titlepage}
\begin{center}
Instytut Informatyki Uniwersytetu Wrocławskiego \\
Studencka Pracownia Inżynierii Oprogramowania, G4 \\
\vspace{4cm}
\Large Michał Kopacz, Mateusz Nahalewicz, Karol Bajko \\
\vspace{0.5cm}
\huge Dokumentacja projektu \mbox{\textbf{WiDz -- Wirtualny Dziennik}} \\ \Large Plan testowania\\
\vspace{1cm}
\normalsize Wersja 1.0
\vfill
\normalsize Wrocław 2010
\end{center}
\end{titlepage}

\newpage

\begin{table}
	\centering
	\caption{Historia zmian w~dokumencie}
		\begin{tabular}{|r|c|c|l|l|}
		\hline
		Lp. 	& Data       & Nr wersji 	& Autorzy           		& Zmiana \\ \hline
		1.   	& 2010-01-10 & 1.0       	& Michał Kopacz & Utworzenie dokumentu \\ \hline
		\end{tabular}
\end{table}

\newpage

\tableofcontents

\newpage

\section{Wprowadzenie}
\subsection{Cel dokumentu}
\noindent Celem dokumentu jest określenie zakresu testów sprawdzających zgodność programu WiDz ze specyfikacją wymagań opisanych w dokumencie \cite{WYM}, oraz wydajność i bezpieczeństwo podczas używania programu. Testy weryfikują również funkcje poszczególnych części programu i~spójność między nimi. Przeprowadzenie testów ma za zadanie zapewnić najwyższą jakość programu WiDz. \\
\subsection{Fazy testów}
\noindent Testowanie programu WiDz przebiega zgodnie z~założeniami modelu V. Zaletą modelu V jest fakt, że dwie najważniejsze fazy: projektowanie i weryfikacja, nie następują po sobie, ale są wykonywane równolegle. \\
\indent Testy będą planowane przed wykonaniem każdej iteracji budowy programu. Będą obejmować sprawdzenie każdego modułu z~osobna, oraz przetestowanie spójności modułów po zakończeniu iteracji. Gotowy program zostanie poddany całościowym testom systemowym po zakończeniu fazy budowy.

\section{Rodzaje testów}
\subsection{Testy modułowe}
\noindent Testy modułowe programu WiDz są projektowane w oparciu o bibliotekę RSpec.\footnote{Biblioteka BDD (ang. BDD --- \textit{Behavior-Driven Development}) dla progarmu napisanego w języku Ruby, umożliwia pisanie testów modułowych} \\

\subsubsection{Testy brzegowe}
\noindent Testy brzegowe pozwalają wykryć i przetestować obszary, w których prawdopodobieństwo niepoprawnego działania programu WiDz jest największe. Podczas testowania zostaną sprawdzone warunki brzegowe powstałe na skutek: 
\begin{itemize}
	\item wprowadzenia całkowicie błędnych lub niespójnych danych wejściowych, 
	\item nieprawidłowo sformatowanych danych, np. adres e-mail wpisywany podczas rejestracji użytkownika w programie WiDz,
	\item pojawienia się wartości znacznie przekraczających oczekiwania, np. ocena wpisywana do dziennika,
	\item pojawienia się duplikatów na listach, np. podczas wyświetlania listy uczniów danej klasy,
	\item zakłócenia przewidywanego porządku zdarzeń, np. próby wygenerowania raportu przed zalogowaniem się.
\end{itemize} 
\subsection{Testy integracyjne}
\subsubsection{Testy funkcjonalności w oparciu o przypadki użycia}
\noindent Testy funkcjonalności i~zgodności ze specyfikacją wymagań programu WiDz, są przeprowadzane za pomocą przypadków testowych, stworzonych na bazie przypadków użycia, określonych w~dokumencie \cite{PU}. \\
\indent Testy sprawdzają działanie programu poprzez interfejs użytkownika. Sprawdzana jest reakcja programu, zarówno na poprawne jak i niepoprawne zachowania użytkownika. \\
\subsubsection{Testy wydajnościowe}
\noindent Podczas testów wydajnościowych programu WiDz zostaną zweryfikowane:
\begin{itemize}
	\item czas odpowiedzi programu przy dużym obciążeniu serwera,
	\item czas odpowiedzi bazy danych przy odczycie dużej ilości danych,
	\item odporność bazy danych na przeciążenie zbyt dużą ilością danych do zapisu.
\end{itemize}
\indent Testy są wykonywane po każdej fazie iteracji, dzięki czemu uzyskana zostaje informacja o~tym, które elementy programu należy poprawić, aby program działał szybciej i stabilniej. \\
\subsubsection{Testy spójności komponentów}
\noindent Testowanie spójności komponentów umożliwia sprawdzenie poprawnego przesyłu danych między komponentami. Sprawdzana jest również spójność danych w bazie danych. \\
\subsubsection{Testy regresywne}
\noindent Testy regresywne są przeprowadzane po każdym zakończeniu iteracji. Maja za zadanie sprawdzić poprawne działanie programu po wprowadzeniu zmian i naprawie błędów.\\
\subsection{Testy systemowe}
\subsubsection{Testy bezpieczeństwa}
\noindent Testy bezpieczeństwa mają za zadanie sprawdzić odporność programu na próby:
\begin{itemize}
	\item włamania do bazy danych metodą SQL-Injection,
	\item zalogowania się na konto użytkownika bez znajomości hasła,
	\item uzyskania przez użytkownika nie przysługujących mu uprawnień,
	\item zablokowania dostępu do programu poprzez przeciążenie programu.
\end{itemize}
\subsubsection{Testy dotyczące interfejsu użytkownika}
\noindent Interfejs użytkownika programu WiDz jest sprawdzany pod względem intuicyjnej nawigacji, szybkości dostępu do poszukiwanej informacji oraz zrozumiałej dla użytkownika komunikacji. \\
\subsubsection{Testy konfiguracyjne}
\noindent Podczas testów sprawdzane jest prawidłowe działanie i~dostępność programu WiDz w~zależności od systemu operacyjnego (Linux, Windows, MacOS), oraz przeglądarki użytkownika (Safari, Firefox, Internet Explorer, Konqueror, Opera, Google Chrome).\\
\subsection{Testy akceptujące}
\noindent Są przeprowadzane przez użytkowników programu WiDz. \\
\section{Zasoby}
\subsection{Testerzy}
\subsection{Sprzęt}
\subsection{Oprogramowanie}
\section{Raporty}
bugzilla
\begin{thebibliography}{9}
	\bibitem{WYM} Michał Kopacz, Karol Bajko, Mateusz Nahalewicz: {\it Dokumentacja projektu WiDz -- Wirtualny Dziennik. Specyfikacja wymagań}. Wrocław, SPIO IIUWr 2009.
	\bibitem{PU} Michał Kopacz, Karol Bajko, Mateusz Nahalewicz: {\it Dokumentacja projektu WiDz -- Wirtualny Dziennik. Analiza przypadków użycia}. Wrocław, SPIO IIUWr 2009.
\end{thebibliography}
\end{document}