\documentclass[12pt,leqno,twoside]{mwart}
\usepackage[utf8x]{inputenc}
\usepackage{polski}
\usepackage{a4wide}
\usepackage{url}

\widowpenalty=10000
\clubpenalty=10000
\raggedbottom

%---------------------------------------------------------------------------------
%paginy!
\usepackage{fancyhdr}
\pagestyle{fancy}
\fancyhead{}
\renewcommand{\sectionmark}[1]{\markright{\thesection.\ #1. }}
\fancyhead[LE]{WiDz -- Wirtualny Dziennik. Specyfikacja wymagań}
\fancyhead[RO]{\rightmark}
\fancyfoot{} % clear all footer fields
\fancyfoot[LE,RO]{}
\fancyfoot[CE]{\thepage}
\fancyfoot[CO]{\thepage}
\addtolength{\headheight}{1.5pt} % pionowy odstep na kreske
\renewcommand{\headrulewidth}{0.4pt}
\renewcommand{\footrulewidth}{0.0pt}
%--------------------------------------------------------------------------------

%\makeatletter
%\renewcommand{\@biblabel}[1]{\quad #1.}
%\makeatother

\renewcommand{\figurename}{Rys.}
\renewcommand{\labelitemi}{-}

\begin{document}

\begin{titlepage}
\begin{center}
Instytut Informatyki Uniwersytetu Wrocławskiego \\
Studencka Pracownia Inżynierii Oprogramowania, G4 \\
\vspace{4cm}
\Large Michał Kopacz, Mateusz Nahalewicz, Karol Bajko \\
\vspace{0.5cm}
\huge Dokumentacja projektu \mbox{\textbf{WiDz -- Wirtualny Dziennik}} \\ \Large Specyfikacja wymagań\\
\vspace{1cm}
\normalsize Wersja 1.3
\vfill
\normalsize Wrocław 2009
\end{center}
\end{titlepage}

\newpage

\begin{table}
	\centering
	\caption{Historia zmian w~dokumencie}
		\begin{tabular}{|r|c|c|l|l|}
		\hline
		Lp. 	& Data       & Nr wersji 	& Autorzy           		& Zmiana \\ \hline
		1.   	& 2009-11-24 & 1.0       	& Michał Kopacz & Utworzenie dokumentu \\ \hline
		2.	& 2009-12-01 & 1.1			& Karol Bajko & Korekta	\\ \hline
		3.	& 2009-12-02 & 1.2			& Michał Kopacz & Korekta	\\ \hline
		4.	& 2009-12-30 & 1.3			& Michał Kopacz & Korekta	\\ \hline
		5.	& 2010-01-04 & 1.4			& Karol Bajko	& Korekta	\\ \hline
		\end{tabular}
\end{table}

\newpage

\tableofcontents

\newpage

\section{Ogólny opis programu WiDz}
\noindent Program WiDz pełni rolę elektronicznego dziennika, który oprócz możliwości tradycyjnego dziennika, takich jak przegląd ocen, rejestrowanie uwag o~zachowaniu i~postawach uczniów, wpisywanie tematów lekcji, dostarcza nowych funkcji ułatwiających pracę nauczyciela i~szkoły. Dzięki temu program WiDz całkowicie zastępuje tradycyjny dziennik\footnote{Zgodnie z~rozporządzeniem Ministra Edukacji Narodowej z~dnia 16 lipca 2009 r. zmieniającym rozporządzenie w~sprawie sposobu prowadzenia przez publiczne przedszkola, szkoły i~placówki dokumentacji przebiegu nauczania, działalności wychowawczej i~opiekuńczej oraz rodzajów tej dokumentacji (Dz. U. 2009, nr 116, poz. 997), które zezwala m.in. na całkowite zastąpienie w~placówce szkolnej dziennika w~formie papierowej na wersję elektroniczną.}, a~dodatkowo usprawnia komunikację między opiekunami a~szkołą. Opiekunowie mają dostęp do informacji o~ocenach i~frekwencji podopiecznych, przeglądając je w~domu, nie muszą w~tym celu udawać się do placówki szkolnej. Dodatkowo są informowani o~ważnych wydarzeniach drogą esemesową oraz mają możliwość regulowania opłat na rzecz szkoły za pośrednictwem Internetu.\\
\indent Raz wprowadzone informacje są wielokrotnie wykorzystywane w~programie, przykładowo podczas generowania raportów ze statystykami o~osiągnięciach w~nauce ucznia, klasy, a~nawet całej szkoły oraz podczas drukowania świadectw.\\
\indent Dostęp do programu WiDz wymaga upoważnienia, a~każdy zapis w~programie  kojarzony jest z~osobą, która go wprowadziła. Program pamięta również historię wprowadzonych zmian, co ma na celu zwiększenie bezpieczeństwa korzystania z~programu.

\section{Charakterystyka użytkowników}\label{CHARAKTER_UZYTK}
\subsection{Uczeń}
\noindent Uczeń ma wgląd jedynie w~swoje dane. Może przeglądać oceny z~przedmiotów, na które uczęszcza, oraz nieobecności na tych przedmiotach.

\subsection{Opiekun}
\noindent Opiekun ma wgląd jedynie w~dane ucznia, nad którym sprawuje opiekę. Może przeglądać oceny z~przedmiotów, na które uczęszcza uczeń oraz nieobecności na tych przedmiotach. Ma również możliwość usprawiedliwiania nieobecności.

\subsection{Nauczyciel}
\noindent Nauczyciel ma wgląd w~dane przedmiotów, które prowadzi. Może dodawać, modyfikować i~usuwać oceny i~nieobecności uczniów w~ramach tych przedmiotów.

\subsection{Wychowawca klasy}
\noindent Wychowawca klasy ma wgląd w~oceny i~nieobecności uczniów swojej klasy. Może dodatkowo usprawiedliwiać nieobecności.

\subsection{Dyrektor}
\noindent Program WiDz dla dyrektora szkoły jest skutecznym narzędziem kontroli systematycznej pracy nauczycieli w~zakresie bieżącego wpisywania ocen i~frekwencji oraz powiadamiania rodziców o~problemach wychowawczych dzieci. Dyrektor szkoły ma wgląd w~dane wszystkich uczniów i~pracowników szkoły mających konto w~programie WiDz.

\section{Wymagania funkcjonalne}
\subsection{Dostęp do programu WiDz}
\subsubsection{Rejestracja użytkownika w~programie}
\noindent Uzyskanie dostępu do programu WiDz wymaga od użytkownika założenia konta, przez wypełnienie formularza rejestracji, a~następnie potwierdzenia tożsamości u administratora. Podczas rejestracji jest wymagane podanie unikatowej nazwy użytkownika i~hasła składającego się z co najmniej 5 znaków, w~tym z co najmniej jednej cyfry. Użytkownik podaje również dane osobowe oraz wyraża zgodę na przechowywanie i~przetwarzanie tych danych\footnote{Zgodnie z~ustawą z~dnia 29 sierpnia 1997 r. o~ochronie danych osobowych (Dz. U. 2002 nr 101 poz.~926).}. Użytkownik ma również prawo wglądu do dotyczących go danych i~ewentualnej ich modyfikacji.

\subsubsection{Potwierdzenie tożsamości przez użytkownika w~programie WiDz}
\noindent Dostęp do funkcji programu WiDz wymaga potwierdzenia tożsamości użytkownika. Po poprawnym potwierdzeniu tożsamości użytkownik uzyskuje dostęp do funkcji zgodnych z~uprawnieniami wymienionymi w~rozdz. \ref{CHARAKTER_UZYTK}. 

\subsection{Osiągnięcia edukacyjne ucznia}
\subsubsection{Dodawanie informacji o~ocenach}
\noindent Nauczyciel ma możliwość wpisania do dziennika w~programie WiDz ocen postępów w~nauce ucznia. Dodając ocenę, może również wybrać zdefiniowany przez niego typ oceny np. zeszyt, sprawdzian. \\
\indent Program umożliwia wpisanie oceny opisowej oraz oceny w~formacie numerycznym według następującej skali\footnote{Skala zgodna z~rozporządzeniem Ministra Edukacji Narodowej z~dnia 30 kwietnia 2007 r. w~sprawie warunków i~sposobu oceniania, klasyfikowania i~promowania uczniów i~słuchaczy oraz przeprowadzania sprawdzianów i~egzaminów w~szkołach publicznych (Dz. U. 2007 nr 83 poz. 562).}: \\
\indent stopień celujący -- 6, \\
\indent stopień bardzo dobry -- 5,\\
\indent stopień dobry -- 4, \\
\indent stopień dostateczny -- 3, \\
\indent stopień dopuszczający -- 2, \\
\indent stopień niedostateczny -- 1. \\

\subsubsection{Uzyskiwanie informacji o~ocenach}
\noindent Program umożliwia przegląd ocen z~dowolnego przedmiotu i~z~dowolnego zakresu czasu.

\subsubsection{Przewidywanie oceny semestralnej ucznia z~każdego przedmiotu}
\noindent Oprócz ocen cząstkowych program wyświetla proponowaną ocenę semestralną z~przedmiotu. Nauczyciel ma możliwość zmiany proponowanej oceny semestralnej.

\subsection{Frekwencja ucznia na lekcji}
\subsubsection{Dodanie informacji o~nieobecności lub spóźnieniu}
\noindent Nauczyciel, sprawdzając obecność uczniów na zajęciach, może dodać w~programie WiDz informacje o~nieobecności ucznia na zajęciach. Jeśli uczeń spóźni się na zajęcia, nauczyciel może zmienić nieobecność na spóźnienie. Dodatkowo program kontroluje przypisania uczniów do zajęć edukacyjnych i~nie pozwala na wpisanie nieobecności uczniom na zajęciach, w~których nie uczestniczą.

\subsubsection{Przegląd frekwencji ucznia}
\noindent WiDz umożliwia przegląd informacji o~frekwencji w~postaci statystyk liczbowych oraz wykresów. Informacje te można dodatkowo ograniczyć do dowolnego zakresu czasu lub przedmiotu.

\subsubsection{Usprawiedliwianie nieobecności}
\noindent Opiekun i~wychowawca klasy mają możliwość usprawiedliwienia nieobecności ucznia na zajęciach.

\subsection{Raporty i~wydruki}
\subsubsection{Drukowanie świadectw} 
\noindent Program WiDz umożliwia drukowanie świadectw szkolnych\footnote{Wzory świadectw zgodne z~załącznikiem do rozporządzenia Ministra Edukacji Narodowej z~dnia 11~marca~2008 r. zmieniające rozporządzenie w~sprawie zasad wydawania oraz wzorów świadectw, dyplomów państwowych i~innych druków szkolnych, sposobu dokonywania ich sprostowań i~wydawania duplikatów, a~także zasad legalizacji dokumentów przeznaczonych do obrotu prawnego z~zagranicą oraz zasad odpłatności za wykonywanie tych czynności (Dz. U. 2008, nr 67, poz. 412).} na giloszu. Świadectwo zostaje automatycznie wypełnione danymi z~programu, można je dowolnie modyfikować oraz wybierać krój i~rozmiar czcionki. Program umożliwia również archiwizację świadectw oraz zapis w~postaci pliku pdf\,\footnote{Dokumety w~formacie pdf można przeglądać programem Adobe Reader.}.

\subsection{Plan lekcji}
\subsubsection{Tworzenie planu lekcji}
\noindent Program umożliwia wprowadzenie planu lekcji szkoły. Sprawdza, czy nie występują w~nim niezgodności, takie jak prowadzenie przez nauczyciela dwóch zajęć w~tym samym czasie, przypisanie klasie dwóch zajęć w~tym samym czasie, brak wymaganej ilości godzin zajęć z~danego przedmiotu w~tygodniu dla danej klasy.\\
\indent Podczas układania planu korzysta się z~wprowadzonych wcześniej informacji o~nauczycielach, przedmiotach i~klasach.

\subsubsection{Przegląd planu lekcji}
\noindent W~WiDz-u jest możliwe przeglądanie planu lekcji całej szkoły, każdego ucznia, nauczyciela i~klasy.

\subsection{Opłaty na rzecz szkoły}
\subsubsection{Umieszczanie informacji o~opłatach}
\noindent W~programie WiDz wychowawca lub sekretariat szkoły mogą dodać informacje o~opłacie za ubezpieczenie, wycieczkę lub inną należność i~określić wysokość tej opłaty. Istnieje również możliwość ustalenia osób, których opłata dotyczy.

\subsubsection{Dokonywanie opłat}
\noindent WiDz wyświetla informacje o~opłatach, przygotowanych przez wychowawcę lub sekretariat, które dotyczą danego ucznia, a~opiekun może je uregulować za pomocą systemu PayPal\footnote{Zob. \url{http://www.paypal.com}.} (p. \ref{PLATNOSCI_WIDZ}). 

\subsection{Komunikacja szkoła--opiekun}
\subsubsection{Powiadamianie opiekuna o~ważnych wydarzeniach drogą esmesową}
\noindent WiDz umożliwia wysłanie esemesem przez wychowawcę, nauczyciela lub sekretariat szkoły ważnej informacji do opiekuna ucznia  (p. \ref{WYSYLANIE_SMS}).

\subsubsection{Wysyłanie informacji do opiekuna pocztą elektroniczną lub za pomocą wiadomości prywatnych}
\noindent Nauczyciel lub wychowawca klasy może za pomocą programu wysłać wiadomość do opiekuna ucznia pod jego adresem elektronicznym, dostępnym w~systemie, lub bezpośrednio na konto użytkownika w~programie WiDz, za pomocą ,,wiadomości prywatne''.

\subsubsection{Umieszczanie informacji na wirtualnej tablicy ogłoszeń}
\noindent Program WiDz umożliwia nauczycielom, wychowawcom, sekretariatowi i~dyrektorowi umieszczanie informacji na wirtualnej tablicy ogłoszeń i~określenie grupy użytkowników, dla której ta wiadomość będzie widoczna.

\subsection{Ocena zajęć}
\subsubsection{Ocenianie zajęć przez ucznia}
\noindent Uczeń ma możliwość ocenienia każdych odbytych zajęć, przez przypisanie im liczby od 1 do 6 wraz z~uzasadnienieniem i~uwagami do zajęć.

\subsubsection{Tworzenie ankiety przez nauczyciela}
\noindent Program umożliwia nauczycielowi tworzenie ankiet do każdego przedmiotu, który prowadzom, w~postaci pytań z~listą gotowych odpowiedzi do wyboru, lub takich na które odpowiedź jest udzielana w~postaci opisowej. Wypełnienie ankiety przez ucznia jest nieobowiązkowe i~anonimowe.

\section{Wymagania niefunkcjonalne}
\subsection{Bezpieczeństwo}
\noindent Program WiDz wprowadza następujące wymagania odnośnie do bezpieczeństwa użytkowania oraz poufności gromadzonych danych:
\begin{itemize}
	\item Po uwierzytelnieniu użytkownika w~programie komunikacja między serwerem a~przeglądarką użytkownika odbywa się za pomocą protokołu TSL 1.1, który szyfruje wysłane dane.
	\item Hasła do konta użytkownika są przechowywane w~bazie danych w~postaci zaszyfrowanej funkcją skrótu SHA-1. Jedyną możliwością odzyskania hasła jest zmiana hasła na nowe.
	\item Dostęp do systemu operacyjnego na serwerze wymaga uwierzytelnienia i~nie jest możliwy z~zewnątrz.
	\item Środowisko ramowe Ruby On Rails zapewnia ochronę przed atakami XSS, CSRF i~SQL Injection.
\end{itemize}

\subsection{Niezawodność}
\noindent Program WiDz jest dostępny 24 godziny na dobę. W~czasie największej aktywności programu, czyli podczas zajęć lekcyjnych, nad prawidłową pracą programu czuwa administrator. Wszelkie uaktualnienia programu powinny odbywać się w~godzinach nocnych.\\
\indent Aby zapobiec utracie danych program codziennie wykonuje kopię bazy danych (w godzinach nocnych).

\subsection{Interfejsy programu}
\subsubsection{Oprogramowanie po stronie serwera}
\noindent Serwer, na którym jest uruchomiona program WiDz, ma zainstalowany system operacyjny Linux. W~systemie operacyjnym jest zainstalowane następujące oprogramowanie: interpreter języka Ruby, menedżer pakietów RubyGems do zarządzania bibliotekami, koniecznymi do uruchomienia programu, oprogramowanie serwerowe HTTP~Apache z~dodatkowymi modułami umożliwiającymi współpracę z~Ruby on Rails, biblioteka kryptograficzna OpenSSL oraz SZBD MySQL.

\subsubsection{Oprogramowanie po stronie klienta}
\noindent Aby skorzystać z~programu WiDz, klient musi mieć sprawne łącze internetowe oraz przeglądarkę obsługującą protokół HTTPS.

\subsection{Interfejsy sprzętowe}
\noindent minimalne wymagania:\\
	\indent procesor: 1 GHz, \\
	\indent RAM: 512 MB, \\
	\indent miejsce na dysku twardym: 10 GB. \\
\noindent \\
\noindent zalecane wymagania: \\
	\indent procesor: 3GHz, dwurdzeniowy, \\
	\indent RAM: 4 GB, \\ 
	\indent miejsce na dysku twardym: 100 GB. \\

\subsection{Wymagania zewnętrzne}
\noindent Wysyłanie esemesów oraz dokonywanie płatności wymaga do poprawnego działania rejestracji lub zakupu usług świadczonych przez inne firmy. W~przypadku braku zakupu lub rejestracji jakiejś usługi program WiDz będzie nadal działać poprawnie, lecz związana z~usługą funkcja będzie nieaktywna.

\subsubsection{Wysyłanie sms-ów}\label{WYSYLANIE_SMS}
\noindent Istnieje wiele polskich i~zagranicznych firm świadczących usługi związane z~wysyłaniem esemesów za pośrednictwem Internetu. Program WiDz używa w~tym celu interfejs API udostępniony przez smsAPI.pl\footnote{Dokumentacja API serwisu smsAPI.pl jest dostępna na stronie \url{http://www.smsapi.pl/smsAPI.pdf}.}. Aby skorzystać z~usług smsAPI.pl, szkoła musi dokonać rejestracji w~witrynie smsAPI.pl oraz zakupić pakiet esemesów\footnote{Zgodnie z~cennikiem obowiązującym na stronie \url{http://www.smsapi.pl/prices.html}.}, a~następnie wprowadzić nazwę użytkownika, hasło oraz typ usługi\footnote{O nazwach ecoSMS lub proSMS.} w~programie WiDz. Zaleca się wykupienie usługi proSMS\footnote{Porównanie usług ecoSMS i~proSMS zawiera dokument \hbox{\url{http://www.smsapi.pl/ecoSMS-proSMS.pdf}}.}, która umożliwia modyfikacje pola nadawcy.

\subsubsection{Dokonywanie płatności za pośrednictwem programu WiDz}\label{PLATNOSCI_WIDZ}
\noindent Program WiDz umożliwia dokonywanie płatności za pomocą usług serwisu PayPal. Integracja w~programie WiDz usług PayPal jest opisana w \cite{PAYPAL}. Od szkoły wymaga się założenia w~witrynie PayPal konta firmowego. Wielkości opłat za transakcje w~przypadku płatności krajowych są dostępne w~cenniku opłat PayPal\footnote{Zob. \url{https://www.paypal.com/pl/cgi-bin/webscr?cmd=_display-receiving-fees-outside&nav=2.0.3}.}.

\section{Słownik}
\noindent Słownik terminów użytych w~dokumencie znajduje się w~\cite{SLO}.

\begin{thebibliography}{9}
	\bibitem{PAYPAL} PayPal Inc.: {\it Płatności za pośrednictwem witryny — podręcznik integracji}, 2006 (\url{https://www.paypalobjects.com/WEBSCR-600-20091109-1/pl_PL/pdf/PP_WebsitePaymentsStandard_IntegrationGuide.pdf}).
	\bibitem{SLO} Michał Kopacz, Karol Bajko, Mateusz Nahalewicz: {\it Dokumentacja projektu WiDz -- Wirtualny Dziennik. Słownik}. Wrocław, SPIO IIUWr 2009.
\end{thebibliography}

\end{document}
