\documentclass[12pt,leqno,twoside]{mwart}
\usepackage[polish]{babel}
\usepackage[utf8]{inputenc}
\usepackage[T1]{fontenc}
\usepackage{a4wide}
\usepackage[titles]{tocloft}
\usepackage{url}

\widowpenalty=10000
\clubpenalty=10000
\raggedbottom

%---------------------------------------------------------------------------------
%paginy!
\usepackage{fancyhdr}
\pagestyle{fancy}
\fancyhead{}
\renewcommand{\sectionmark}[1]{\markright{\thesection.\ #1. }}
\renewcommand{\sectionmark}[1]{\markright{\thesection.\ #1}}
\fancyhead[LE]{WiDz -- Wirtualny Dziennik. Architektura oprogramowania}
\fancyhead[RO]{\rightmark}
\fancyfoot{} % clear all footer fields
\fancyfoot[LE,RO]{}
\fancyfoot[CE]{\thepage}
\fancyfoot[CO]{\thepage}
\addtolength{\headheight}{1.5pt} % pionowy odstep na kreske
\renewcommand{\headrulewidth}{0.4pt}
\renewcommand{\footrulewidth}{0.0pt}
%--------------------------------------------------------------------------------

%\makeatletter
%\renewcommand{\@biblabel}[1]{\quad #1.}
%\makeatother

\renewcommand{\figurename}{Rys.}
\renewcommand{\labelitemi}{-}

%\makeatletter
%\renewcommand{\@pnumwidth}{1.75em}
%\renewcommand{\@tocrmarg}{2.75em}
%\makeatother

%\setlength{\cftbeforechapskip}{2ex}
%\setlength{\cftbeforesecskip}{0.5ex}

\begin{document}

\begin{titlepage}
\begin{center}
Instytut Informatyki Uniwersytetu Wrocławskiego \\
Studencka Pracownia Inżynierii Oprogramowania, G4 \\
\vspace{4cm}
\Large Michał Kopacz, Mateusz Nahalewicz, Karol Bajko \\
\vspace{0.5cm}
\huge Dokumentacja projektu \mbox{\textbf{WiDz -- Wirtualny Dziennik}} \\ \Large Architektura oprogramowania\\
\vspace{1cm}
\normalsize Wersja 1.0
\vfill
\normalsize Wrocław 2009
\end{center}
\end{titlepage}

\newpage
\vfill
\begin{table}[tb]
	\centering
	\caption{Historia zmian w~dokumencie}
		\begin{tabular}{|r|c|c|p{5,5cm}|l|}
		\hline
		Lp. 	& Data       & Nr wersji 	& Autor           		& Zmiana \\ \hline
		1.   	& 2009-12-08 & 1.0       	& \mbox{Karol Bajko} & Utworzenie dokumentu \\ \hline
		\end{tabular}
\end{table}

\tableofcontents
\newpage

\section{Wstęp}
\noindent Niniejszy dokument ma na celu przedstawienie czytelnikowi wizji architektury aplikacji WiDz. Prezentuje on jej kluczowe składniki, a także skupia się na analizie architektury z~kilku perspektyw -- logicznej, implementacyjnej, wdrożeniowej, procesowej i przypadków użycia. Wszystko po to by ułatwić potencjalnemu klientowi zrozumienie kwestii architektonicznych WiDz.\\
\indent Opisana poniżej architektura może ulec zmianie w~fazie implementacji, lecz ogólny schemat aplikacji nie powinien zostać naruszony.

\section{Elementy architektury WiDz}
\subsection{Serwer}
\noindent Serwer jest podstawowym składnikiem architektury, gdyż na nim zostanie uruchomiona aplikacja WiDz. Posiada on zainstalowany system operacyjny, oparty o jądro Linux, a w nim dostępny interpreter języka Ruby, menadżer pakietów RubyGems do zarządzania bibliotekami, koniecznymi do uruchomienia aplikacji, oprogramowanie serwerowe HTTP Apache z~modułem Passenger -- niezbędny do współpracy z~aplikacją stworzoną w~Ruby on Rails, biblioteka kryptograficzna OpenSSL oraz SZBD MySQL.\\
\indent Serwer jest odpowiedzialny za umożliwienie ciągłego dostępu do aplikacji każdemu użytkownikowi, bez względu na porę dnia. Odpowiada za okresowe tworzenie kopii bezpieczeństwa danych zawartych w~bazie danych. 

\subsection{Baza danych}
\noindent Baza danych jest kluczowym elementem aplikacji, ponieważ przechowuje wszystkie dane związane z oferowanymi usługami aplikacji WiDz. Zawiera dane nauczycieli, opiekunów i uczniów placówki szkolnej, wraz z informacjami o postępach w nauce, frekwencji. Przetrzymuje dane o dokonanych opłatach, obecnym harmonogramie zajęć w placówce oraz archiwizuje wszelkie wiadomości przesyłane pomiędzy użytkownikami WiDz.\\
\indent Baza danych  MySQL jest umieszczona na serwerze. Korzystać z niej może jedynie aplikacja WiDz, która będzie miała bezpośredni dostęp do danych w niej przetrzymywanych. Komunikacja pomiędzy aplikacją a SZBD MySQL opiera się na języku zapytań SQL.

\subsection{Pliki źródłowe}
\noindent Kod źródłowy WiDz został napisany w języku Ruby przy wykorzystaniu ramy projektowej Ruby on Rails. Język ten okazał się najlepszy do stworzenia aplikacji, ponieważ oferuje szybkie, łatwe tworzenie rozbudowanych projektów, umożliwiając przy tym większe skupienie na samych funkcjonalnościach, które ma oferować, sprowadzając do minimum konieczność niskopoziomowego kodowania, gdyż Ruby on Rails wprowadza mnóstwo gotowych, domyślnych konwencji i bibliotek do wykorzystania\footnote{zasada ''Konwencja ponad Konfigurację'' (ang. \textit{Convention over configuration})}. Sama aplikacja została stworzona przy wykorzystaniu wzorca projektowego MVC, które wydzieliło aplikację na trzy części:
\begin{itemize}
\item warstwa modelu odpowiedzialna za komunikację z bazą danych i przetwarzanie danych,
\item warstwa widoku odpowiedzialna za prezentowanie wyników działania aplikacji na ekranie monitora,
\item warstwa kontrolera odpowiedzialna za przechwytywanie żądań użytkownika i na ich podstawie generowanie odpowiedzi, wykorzystując dwie powyższe warstwy.
\end{itemize}
Jak widać, pliki źródłowe pełnią krytyczną rolę w aplikacji. Są one najważniejszą częścią całego projektu, ponieważ to one są odpowiedzialne za sprawną współpracę bazy danych i użytkowników korzystającym z WiDz.

\subsection{Interfejs użytkownika}
\noindent Interfejs użytkownika to element bardzo istotny w całym projekcie, którego sukces zależy w dużym stopniu od intuicyjności, przejrzystości i łatwości w obsłudze. Języki, które wykorzystano w celu zaprojektowania przyjaznego interfejsu to HTML, CSS, JavaScript. 

\section{Perspektywy architektoniczne}
\subsection{Perspektywa logiczna}
\subsection{Perspektywa implementacyjna}
\subsection{Perspektywa wdrożeniowa}
\subsection{Perspektywa procesowa}
\subsection{Perspektywa przypadków użycia}

\section{Słownik}
\begin{itemize}
\item CSS
\item HTML
\item JavaScript
\item MVC
\end{itemize}
\noindent Słownik terminów użytych w~dokumencie znajduje się w~\cite{SLO}.

\begin{thebibliography}{99}
\bibitem{SLO} Michał Kopacz, Mateusz Nahalewicz, Karol Bajko, {\it Dokumentacja projektu WiDz -- Wirtualny Dziennik. Słownik}. Wrocław, SPIO IIUWr 2009.
\end{thebibliography}

\end{document}
