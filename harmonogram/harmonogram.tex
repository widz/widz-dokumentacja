\documentclass[12pt,leqno,twoside]{mwart}
\usepackage[utf8x]{inputenc}
\usepackage{polski}
\usepackage{a4wide}
\usepackage{url}
\usepackage{array}
\widowpenalty=10000
\clubpenalty=10000
\raggedbottom

%---------------------------------------------------------------------------------
%paginy!
\usepackage{fancyhdr}
\pagestyle{fancy}
\fancyhead{}
\renewcommand{\sectionmark}[1]{\markright{\thesection.\ #1. }}
\fancyhead[LE]{WiDz -- Wirtualny Dziennik. Harmonogram}
\fancyhead[RO]{\rightmark}
\fancyfoot{} % clear all footer fields
\fancyfoot[LE,RO]{}
\fancyfoot[CE]{\thepage}
\fancyfoot[CO]{\thepage}
\addtolength{\headheight}{1.5pt} % pionowy odstep na kreske
\renewcommand{\headrulewidth}{0.4pt}
\renewcommand{\footrulewidth}{0.0pt}
%--------------------------------------------------------------------------------

%\makeatletter
%\renewcommand{\@biblabel}[1]{\quad #1.}
%\makeatother

\renewcommand{\figurename}{Rys.}
\renewcommand{\labelitemi}{-}

\begin{document}

\begin{titlepage}
\begin{center}
Instytut Informatyki Uniwersytetu Wrocławskiego \\
Studencka Pracownia Inżynierii Oprogramowania, G4 \\
\vspace{4cm}
\Large Michał Kopacz, Mateusz Nahalewicz, Karol Bajko \\
\vspace{0.5cm}
\huge Dokumentacja projektu \mbox{\textbf{WiDz -- Wirtualny Dziennik}} \\ \Large Harmonogram\\
\vspace{1cm}
\normalsize Wersja 1.1
\vfill
\normalsize Wrocław 2010
\end{center}
\end{titlepage}

\newpage

\begin{table}
	\centering
	\caption{Historia zmian w~dokumencie}
		\begin{tabular}{|r|c|c|l|l|}
		\hline
		Lp. 	& Data       & Nr wersji 	& Autorzy           		& Zmiana \\ \hline
		1.   	& 2010-01-09 & 1.0       	& \mbox{Mateusz Nahalewicz} & Utworzenie dokumentu \\ \hline
		2.   	& 2010-01-10 & 1.1       	& \mbox{Michał Kopacz} & Korekta \\ \hline
		\end{tabular}
\end{table}

\newpage

\tableofcontents

\newpage


\section{Informacje ogólne}
\noindent Budowanie programu WiDz podzielono na pięć etapów, które opisano poniżej. Zespół pracuje po osiem godzin dziennie przez pięć dni w tygodniu (od poniedziałku do piątku). Każdej fazie realizacji przedsięwzięcia przypisano datę jej rozpoczęcia oraz zakończenia. Czas realizacji projektu jest szacowany na około 5,5 miesiąca.\\


\section{Etap analizy}
\noindent Etap analizy składa się z ustalenia wymagań projektu, potrzeb użytkowników oraz podstawowych przypadków użycia, a także wyboru architektury sprzętowej i technologii w jakiej będzie tworzony program WiDz. Czas trwania etapu wynosi 8 dni roboczych.\\

	%\centering
		\begin{tabular}{m{8cm} c c}
		
		& \textbf{Data rozpoczęcia} & \textbf{Data zakończenia} \\ 
		Analiza wymagań & 2009-10-12 & 2009-10-21 \\ 
		\textbf{Ogółem} & \textbf{2009-10-12} & \textbf{2009-10-21} \\ 
		\end{tabular}

\section{Etap projektowania}
\noindent Etap projektowania składa się z zaprojektowania struktury klas i bazy danych, sporządzenia dokładnej specyfikacji architektury sprzętowej, określenia związków między klasami oraz interfejsu użytkownika. Czas trwania etapu wynosi 26 dni roboczych.\\

	%\centering
		\begin{tabular}{m{8cm} c c}
		
		& \textbf{Data rozpoczęcia} & \textbf{Data zakończenia} \\ 
		Projekt klas & 2009-10-22 & 2009-11-06 \\ 
		Projekt bazy danych & 2009-11-09 & 2009-11-11 \\ 
		Powiązanie klas & 2009-11-12 & 2008-11-17 \\ 
		Specyfikacja sprzętu & 2009-11-18 & 2009-11-20 \\ 
		Projekt interfejsu & 2009-11-23 & 2009-11-26 \\ 
		\textbf{Ogółem} & \textbf{2009-10-22} & \textbf{2009-11-26} \\ 
		\end{tabular}

\section{Etap implementacji}
\noindent Etap implementacji składa się z 10 iteracji. W trakcie kolejnych iteracji będą implementowane odpowiednie klasy i funkcje.\\

	%\centering
		\begin{tabular}{m{8cm} c c}
		
		& \textbf{Data rozpoczęcia} & \textbf{Data zakończenia} \\ 
		System uwierzytelniania & 2009-11-27 & 2009-12-18 \\ 
		Zakładanie dziennika & 2009-12-21 & 2010-01-08 \\ 
		Modyfikowanie danych w dzienniku & 2010-01-11 & 2010-01-14 \\
		Przeglądanie dziennika & 2010-01-15 & 2010-01-22 \\ 
		Wprowadzanie i oglądanie planu zajęć & 2010-01-25 & 2010-01-28 \\ 		
		System statystyk & 2010-01-29 & 2010-02-08 \\ 
		System powiadomień & 2010-02-09 & 2010-02-19 \\ 
		System oceny zajęć & 2010-02-22 & 2010-02-26 \\ 
		System płatności & 2010-03-01 & 2010-03-09 \\ 
		Kącik klasowy & 2010-03-10 & 2010-03-12 \\		
		
		\textbf{Ogółem} & \textbf{2009-11-27} & \textbf{2010-03-12} \\ 
		\end{tabular}

\section{Etap testowania}

\noindent Testowanie programu WiDz przebiega zgodnie z założeniami modelu V. Zaletą modelu V
jest fakt, że dwa najważniejsze etapy tj. implementacja i testowanie, nie następują po sobie,
ale są wykonywane równolegle.\\

	%\centering
		\begin{tabular}{m{8cm} c c}
		
		& \textbf{Data rozpoczęcia} & \textbf{Data zakończenia} \\ 
		System uwierzytelniania & 2009-11-27 & 2010-03-16 \\ 
		Zakładanie dziennika & 2009-12-21 & 2010-03-16 \\ 
		Modyfikowanie danych w dzienniku & 2010-01-11 & 2010-03-16 \\
		Przeglądanie dziennika & 2010-01-15 & 2010-03-16 \\ 
		Wprowadzanie i oglądanie planu zajęć & 2010-01-25 & 2010-03-16 \\ 		
		System statystyk & 2010-01-29 & 2010-03-16 \\ 
		System powiadomień & 2010-02-09 & 2010-03-16 \\ 
		System oceny zajęć & 2010-02-22 & 2010-03-16 \\ 
		System płatności & 2010-03-01 & 2010-03-16 \\ 
		Kącik klasowy & 2010-03-10 & 2010-03-16 \\		
		
		\textbf{Ogółem} & \textbf{2009-11-27} & \textbf{2010-03-16} \\ 
		\end{tabular}

\section{Etap przekazania}
\noindent Pierwszym krokiem w trakcie tego etapu jest zainstalowanie serwera, na którym będzie działał program WiDz. Następnie program WiDz jest instalowany na serwerze.\\

	%\centering
		\begin{tabular}{m{8cm} c c}
		
		& \textbf{Data rozpoczęcia} & \textbf{Data zakończenia} \\ 
		Zainstalowanie serwera & 2010-03-17 & 2010-03-22 \\ 
		Zainstalowanie programu WiDz & 2010-03-23 & 2010-03-26 \\ 
		
		\textbf{Ogółem} & \textbf{2010-03-17} & \textbf{2010-03-26} \\ 
		\end{tabular}

\section{Oddanie systemu}
\noindent Planowanym dniem oddania programu WiDz do użytku jest 29 marca 2010.\\


\end{document}
